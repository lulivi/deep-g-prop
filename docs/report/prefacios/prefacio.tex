\thispagestyle{empty}

\begin{center}

    {\large\bfseries Deep G. Prop. \\ Optimización de parámetros de perceptrones
multicapa}\\

\end{center}

\begin{center}

    Luis Liñán Villafranca \\

\end{center}

\vspace{0.7cm}

\vspace{0.5cm}
\noindent{\textbf{Palabras clave}:
\textit{Software libre},
\textit{Redes neuronales},
\textit{Algoritmos genéticos},
\textit{Optimización de prarámetros},
\textit{Clasificación}}
\vspace{0.7cm}

\noindent{\textbf{Resumen}}\\

Existe un problema que concierne a los modelos de clasificación supervisada:
encontrar la configuración de parámetros adecuada al problema al que se aplica.
Los perceptrones multicapa (MLP) con propagación hacia atrás (backpropagation)
tienen multitud valores de entrada, como el número de capas ocultas, el número
de neuronas (o perceptrones) en cada capa, los pesos entre las neuronas, la
función de activación de cada neurona... Lo cual hace que su correcta
inicialización suponga una tarea difícil y costosa. Los algoritmos genéticos
(GA) son una buena opción para solventar ésta, ya que realizando las
operaciones características de éstos (que simulan el proceso de selección
natural) se puede obtener una solución optimizada de parámetros que sirvan
punto de partida a los perceptrones multicapa.

\cleardoublepage

\begin{center}

    {\large\bfseries Deep G. Prop. \\ Multilayer perceptrons parameters
    optimization}\\

\end{center}

\begin{center}

    Luis Liñán Villafranca\\

\end{center}

\vspace{0.5cm}
\noindent{\textbf{Keywords}:
\textit{free software}
\textit{neural networks},
\textit{genetic algorithms},
\textit{parameters optimization},
\textit{classification}}
\vspace{0.7cm}

\noindent{\textbf{Abstract}}\\


\cleardoublepage

\thispagestyle{empty}

\noindent\rule[-1ex]{\textwidth}{2pt}\\[4.5ex]

D. \textbf{Juan Julián Merelo Guervós}, Profesor(a) del ...

\vspace{0.5cm}

\textbf{Informo:}

\vspace{0.5cm}

Que el presente trabajo, titulado \textit{\textbf{Deep G. Prop.}}, ha sido realizado
bajo mi supervisión por \textbf{Luis Liñán Villafranca}, y autorizo la defensa de dicho
trabajo ante el tribunal que corresponda.

\vspace{0.5cm}

Y para que conste, expiden y firman el presente informe en Granada a Noviembre de
2019.

\vspace{1cm}

\textbf{El/la director(a)/es:}

\vspace{5cm}

\noindent\textbf{Juan Julián Merelo Guervós}

\chapter*{Agradecimientos}




