%%%%%%%%%%%%%%%%%%%%%%%%%%%%%%%%%%%%%%%%%
% Short Sectioned Assignment LaTeX Template Version 1.0 (5/5/12)
% This template has been downloaded from: http://www.LaTeXTemplates.com
% Original author:  Frits Wenneker (http://www.howtotex.com)
% License: CC BY-NC-SA 3.0 (http://creativecommons.org/licenses/by-nc-sa/3.0/)
%%%%%%%%%%%%%%%%%%%%%%%%%%%%%%%%%%%%%%%%%

% \documentclass[paper=a4, fontsize=11pt]{scrartcl}
\documentclass[11pt, a4paper]{book}
% Use 8-bit encoding that has 256 glyphs
\usepackage[T1]{fontenc}
\usepackage[utf8]{inputenc}
% Use the Adobe Utopia font for the document - comment this line to return to
% the LaTeX default
\usepackage{fourier}
% para insertar código con formato similar al editor
\usepackage{listings}
% Selecciona el español para palabras introducidas automáticamente, p.ej.
% "septiembre" en la fecha y especifica que se use la palabra Tabla en vez de
% Cuadro
\usepackage[spanish, es-tabla]{babel}
% ,href} %para incluir URLs e hipervínculos dentro del texto (aunque hay que
% instalar href)
\usepackage{url}
% para incluir imágenes y colocarlas
\usepackage{graphics,graphicx, float}
\graphicspath{{../}{./}}
% para incluir el símbolo del euro
\usepackage[gen]{eurosym}
% para incluir citas del archivo <nombre>.bib
\usepackage{cite}
\usepackage{enumerate}
\usepackage{hyperref}
\usepackage{graphicx}
\usepackage{tabularx}
\usepackage{multirow}
\usepackage{booktabs}

\newcommand{\mc}[3]{\multicolumn{#1}{#2}{#3}}
\newcommand{\mr}[3]{\multirow{#1}{#2}{#3}}
\newcommand{\nl}{\tabularnewline}

\usepackage[table,xcdraw]{xcolor}
\hypersetup{
    colorlinks=true,    % false: boxed links; true: colored links
    linkcolor=black,    % color of internal links
    urlcolor=cyan       % color of external links
}
\renewcommand{\familydefault}{\sfdefault}
% Custom headers and footers
\usepackage{fancyhdr}
% Makes all pages in the document conform to the custom headers and footers
\pagestyle{fancyplain}
% Empty left header
\fancyhead[L]{}
% Empty center header
\fancyhead[C]{}
% My name
\fancyhead[R]{Luis Liñán Villafranca}
% Empty left footer
\fancyfoot[L]{}
% Empty center footer
\fancyfoot[C]{}
% Page numbering for right footer
\fancyfoot[R]{\thepage}
% Remove header underlines
%\renewcommand{\headrulewidth}{0pt}
% Remove footer underlines
\renewcommand{\footrulewidth}{0pt}
% Customize the height of the header
\setlength{\headheight}{13.6pt}

\usepackage{titlesec, blindtext, color}
\definecolor{gray75}{gray}{0.75}
\newcommand{\hsp}{\hspace{20pt}}
\titleformat{\chapter}[hang]{%
    \Huge\bfseries%
}{\thechapter\hsp\textcolor{gray75}{|}\hsp}{0pt}{\Huge\bfseries}
\setcounter{secnumdepth}{4}
\usepackage[Lenny]{fncychap}
\usepackage{minted}
\usepackage{tikz}
\usetikzlibrary{babel,positioning}
\usepackage{calc}
\usepackage{caption}

\begin{document}

    % Plantilla portada UGR
    \begin{titlepage}
\newlength{\centeroffset}
\setlength{\centeroffset}{-0.5\oddsidemargin}
\addtolength{\centeroffset}{0.5\evensidemargin}
\thispagestyle{empty}

\noindent\hspace*{\centeroffset}\begin{minipage}{\textwidth}

\centering
\includegraphics[width=0.9\textwidth]{logos/logo_ugr.jpg}\\[1.4cm]

\textsc{ \Large TRABAJO FIN DE GRADO\\[0.2cm]}
\textsc{ GRADO EN INGENIERÍA INFORMÁTICA}\\[1cm]

{\Huge\bfseries Deep G. Prop. \\}
\noindent\rule[-1ex]{\textwidth}{3pt}\\[3.5ex]
{\large\bfseries Optimización de parámetros de perceptrones multicapa}
\end{minipage}

\vspace{2.5cm}
\noindent\hspace*{\centeroffset}
\begin{minipage}{\textwidth}
\centering

\textbf{Autor}\\ {Luis Liñán Villafranca}\\[2.5ex]
\textbf{Director}\\ {Juan Julián Merelo Guervós}\\[2cm]
\includegraphics[width=0.3\textwidth]{logos/etsiit_logo.png}\\[0.1cm]
\textsc{Escuela Técnica Superior de Ingenierías Informática y de
Telecomunicación}\\
\textsc{---}\\
Granada, Julio de 2020
\end{minipage}
\end{titlepage}


    % Plantilla prefacio UGR
    \thispagestyle{empty}

\begin{center}

    {\large\bfseries Deep G. Prop. \\ Optimización de parámetros de perceptrones
multicapa}\\

\end{center}

\begin{center}

    Luis Liñán Villafranca \\

\end{center}

\vspace{0.7cm}

\vspace{0.5cm}
\noindent{\textbf{Palabras clave}:
\textit{Software libre},
\textit{Redes neuronales},
\textit{Algoritmos genéticos},
\textit{Optimización de prarámetros},
\textit{Clasificación}}
\vspace{0.7cm}

\noindent{\textbf{Resumen}}\\

Existe un problema que concierne a los modelos de clasificación supervisada:
encontrar la configuración de parámetros adecuada al problema al que se aplica.
Los perceptrones multicapa (MLP) con propagación hacia atrás (backpropagation)
tienen multitud valores de entrada, como el número de capas ocultas, el número
de neuronas (o perceptrones) en cada capa, los pesos entre las neuronas, la
función de activación de cada neurona... Lo cual hace que su correcta
inicialización suponga una tarea difícil y costosa. Los algoritmos genéticos
(GA) son una buena opción para solventar ésta, ya que realizando las
operaciones características de éstos (que simulan el proceso de selección
natural) se puede obtener una solución optimizada de parámetros que sirvan
punto de partida a los perceptrones multicapa.

\cleardoublepage

\begin{center}

    {\large\bfseries Deep G. Prop. \\ Multilayer perceptrons parameters
    optimization}\\

\end{center}

\begin{center}

    Luis Liñán Villafranca\\

\end{center}

\vspace{0.5cm}
\noindent{\textbf{Keywords}:
\textit{free software}
\textit{neural networks},
\textit{genetic algorithms},
\textit{parameters optimization},
\textit{classification}}
\vspace{0.7cm}

\noindent{\textbf{Abstract}}\\


\cleardoublepage

\thispagestyle{empty}

\noindent\rule[-1ex]{\textwidth}{2pt}\\[4.5ex]

D. \textbf{Juan Julián Merelo Guervós}, Profesor(a) del ...

\vspace{0.5cm}

\textbf{Informo:}

\vspace{0.5cm}

Que el presente trabajo, titulado \textit{\textbf{Deep G. Prop.}}, ha sido realizado
bajo mi supervisión por \textbf{Luis Liñán Villafranca}, y autorizo la defensa de dicho
trabajo ante el tribunal que corresponda.

\vspace{0.5cm}

Y para que conste, expiden y firman el presente informe en Granada a Noviembre de
2019.

\vspace{1cm}

\textbf{El/la director(a)/es:}

\vspace{5cm}

\noindent\textbf{Juan Julián Merelo Guervós}

\chapter*{Agradecimientos}






    % Índice de contenidos
    \newpage
    \tableofcontents

    % Índice de imágenes y tablas
    \newpage
    \listoffigures

    % Si hay suficientes se incluirá dicho índice
    \listoftables
    \newpage

    % Introducción
    \input{secciones/01_introduccion}

    % Descripción del problema y hasta donde se llega
    \input{secciones/02_descripcion}

    % Estado del arte
    % 1. Crítica al estado del arte
    % 2. Propuesta
    \input{secciones/03_estado_del_arte}

    % Planificación del trabajo
    \input{secciones/04_planificacion}

    % Análisis del problema
    % 1. Análisis de requisitos
    % 2. Análisis de las soluciones
    % 3. Solucion propuesta
    % 4. Análisis de seguridad
    \input{secciones/05_analisis}

    % Desarrollo bajo sprints:
    % 1. Permitir registros y login de usuarios
    % 2. Desarrollo del sistema de incidencias
    % 3. Desarrollo del sistema de denuncias administrativas y accidentes
    % 4. Desarrollo del sistema de croquis
    % 5. Instalación de la aplicación de manera automática
    \input{secciones/06_implementacion}

    % Presupuesto

    % Conclusiones
    \input{secciones/07_conclusiones}

    % Trabajos futuros

    \newpage
    \bibliography{bibliografia}
    \bibliographystyle{plain}

\end{document}
